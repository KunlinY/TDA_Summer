\documentclass[12pt,a4paper]{article}
\usepackage{amsmath,amsxtra,amsthm,amssymb,makeidx,graphics,graphicx}
\theoremstyle{plain}
\newtheorem{theorem}{Theorem}[section]
\newtheorem{lemma}[theorem]{Lemma}
\newtheorem{question}[theorem]{Question}
\newtheorem{exercise}[theorem]{Exercise}
\newtheorem{example}[theorem]{Example}
\newtheorem{definition}[theorem]{Definition}
\newtheoremstyle{citing}{3pt}{3pt}{\itshape}{0pt}{\bfseries}%
{.}{ }{\thmnote{#3}}\theoremstyle{citing}\newtheorem*{varthm}{}

\begin{document}

\title{Tutorial on Topological Data Analysis}
\author{Kunlin}
\maketitle

\section{Metric Topology}

\subsection{Motivation}

\textbf{Question: } How to do "\emph{smooth deformation}" of an object. 

\begin{definition}
    Smooth deformation will not change the shape of an object. 
\end{definition}
If we define a shape by counting the number of holes in the object\dots

You cannot change a ball into a donut, because a ball has no holes a donut has one hole. 

In TDA:
\begin{itemize}
    \item Start from a data set
    \item Construct topological objects
    \item Study their properties that are not changed by smooth deformation
\end{itemize}

Challenges: 
\begin{itemize}
    \item How to define shapes (number of holes)?
    \item How to count holes when dimension $n=2,3,4,\cdots$
\end{itemize}

Idea: dimension reduction while maintaining homology. 

\subsection{Open Sets}

\begin{definition}
    Let $(X,d)$ be a metric space. For each element $a \in X$ and each $r\in (0, \infty) \subseteq \mathbb{R}$, the \emph{open ball with a center $a$ and radius $r$} is the subset 
    \begin{equation}
        B_{a,r}=\{x\in X | d(a,x) < r\} \subseteq X
    \end{equation}
\end{definition}

\begin{example}
    Let $(X,d)$ be a metric space. If $r>0$, then 
\begin{equation}
    B_{a,r}=\{x\in X | d(a,x) < r\} \subseteq X
\end{equation}
is open. 
\end{example}
    
\end{document}