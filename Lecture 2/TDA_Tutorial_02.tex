\documentclass[12pt,a4paper]{article}
\usepackage{amsmath,amsxtra,amsthm,amssymb,makeidx,graphics,graphicx}
\theoremstyle{plain}
\newtheorem{theorem}{Theorem}[section]
\newtheorem{lemma}[theorem]{Lemma}
\newtheorem{question}[theorem]{Question}
\newtheorem{exercise}[theorem]{Exercise}
\newtheorem{example}[theorem]{Example}
\newtheorem{definition}[theorem]{Definition}
\newtheoremstyle{citing}{3pt}{3pt}{\itshape}{0pt}{\bfseries}%
{.}{ }{\thmnote{#3}}\theoremstyle{citing}\newtheorem*{varthm}{}

\begin{document}

\title{Tutorial on Topological Data Analysis}
\author{Kunlin}
\maketitle

\section{Metric Topology}

\subsection{Motivation}

\textbf{Question: } How to do "\emph{smooth deformation}" of an object. 

\begin{definition}
    Smooth deformation will not change the shape of an object. 
\end{definition}
If we define a shape by counting the number of holes in the object\dots

You cannot change a ball into a donut, because a ball has no holes a donut has one hole. 

In TDA:
\begin{itemize}
    \item Start from a data set
    \item Construct topological objects
    \item Study their properties that are not changed by smooth deformation
\end{itemize}

Challenges: 
\begin{itemize}
    \item How to define shapes (number of holes)?
    \item How to count holes when dimension $n=2,3,4,\cdots$
\end{itemize}

Idea: dimension reduction while maintaining homology. 

\subsection{Open Sets}

\begin{definition}
    Let $(X,d)$ be a metric space. For each element $a \in X$ and each $r\in (0, \infty) \subseteq \mathbb{R}$, the \emph{open ball with a center $a$ and radius $r$} is the subset 
    \begin{equation}
        B_{a,r}=\{x\in X | d(a,x) < r\} \subseteq X
    \end{equation}
\end{definition}

\begin{example}
    Let $(X,d)$ be a metric space. If $r>0$, then 
\begin{equation}
    B_{a,r}=\{x\in X | d(a,x) < r\} \subseteq X
\end{equation}
is open. 
\end{example}
\noindent\emph{Solution. }
If $y\in B(a,r)$, then 
$\delta=r-d(a,y)>0$ and,
whenever $d(z,y)<\delta$,
the triangle inequality gives us
\[d(a,z)\leq d(a,y)+d(y,z)<r\]
so $z\in B(a,r)$. Thus $B(a,r)$ is open.

\begin{theorem}
    If $(X,d)$ is a metric space, then the following statements are true.

(i) The empty set $\emptyset$ and the space $X$ are open.

(ii) If $U_{\alpha}$ is open for all $\alpha\in A$, then $\bigcup_{\alpha\in A} U_{\alpha}$ is open. (In other words, the union of open sets is open.)

(iii) If $U_{j}$ is open for all $1\leq j\leq n$, then $\bigcap_{j=1}^{n} U_{j}$ is open.
\end{theorem}
\begin{proof}
    (i) Since there are no points $e$ in $\emptyset$, the statement
\[x\in \emptyset\ \text{whenever}\ d(x,e)<1\]
holds for all $e\in \emptyset$. Since every point $x$ belongs to $X$,
the statement
\[x\in X\ \text{whenever}\ d(x,e)<1\]
holds for all $e\in X$.

(ii) If $e\in \bigcup_{\alpha\in A} U_{\alpha}$, then we can find
a particular $\alpha_{1}\in A$ with $e\in U_{\alpha_{1}}$.
Since $U_{\alpha_{1}}$ is open, we can find a $\delta>0$
such that
\[x\in U_{\alpha_{1}}\ \text{whenever}\ d(x,e)<\delta.\]
Since $U_{\alpha_{1}}\subseteq \bigcup_{\alpha\in A} U_{\alpha}$,
\[x\in \bigcup_{\alpha\in A} U_{\alpha}  \text{whenever}\ d(x,e)<\delta.\]
Thus $\bigcup_{\alpha\in A} U_{\alpha}$ is open.

(iii) If $e\in \bigcap_{j=1}^{n} U_{j}$, then
$e\in U_{j}$ for each $1\leq j\leq n$. Since
$U_{j}$ is open, we can find a $\delta_{j}>0$
such that
\[x\in U_{j}\ \text{whenever}\ d(x,e)<\delta_{j}.\]
Setting $\delta=\min_{1\leq j\leq n}\delta_{j}$, we
have $\delta>0$ and
\[x\in U_{j}\ \text{whenever}\ d(x,e)<\delta\]
for all $1\leq j\leq n $. Thus
\[x\in \bigcap_{j=1}^{n} U_{j}\ \text{whenever}\ d(x,e)<\delta\]
and we have shown that $\bigcap_{j=1}^{n} U_{j}$ is open.
\end{proof}

\begin{example}
    Here is an example of how property (iii) faiuls when we consider the intersection of an infinite family of open subsets. Let $X=\mathbb{R}$ with the Euclidean metric, $d(x,y)=|x-y|$, and let $U_i=(-\frac{1}{i},\frac{1}{i})$ for all $i>1$. Then, $\bigcap_{i=1}^\infty U_i=\{0\}$, which is not open. 
\end{example}

\begin{example}
    If we work in ${\mathbb R}^{n}$ with the Euclidean metric, then the one point set $\{{\mathbf x}\}$ is not open.
\end{example}
\begin{proof}
    Choose ${\mathbf e}\in{\mathbb R}^{n}$ with $\|{\mathbf e}\|_{2}=1$.
(We could take ${\mathbf e}=(1,0,0,\ldots,0)$.) If $\delta>0$,
then, setting ${\mathbf y}={\mathbf x}+(\delta/2){\mathbf e}$, we have
$\|{\mathbf x}-{\mathbf y}\|_{2}<\delta$, 
yet ${\mathbf y}\notin\{{\mathbf x}\}$.
Thus $\{{\mathbf x}\}$ is not open.
\end{proof}

\subsection{Continuous functions between metric spaces}
\begin{definition}
    Lext $(X,d_X)$ and $(Y,d_Y)$ be metric spaces. A map $f: X\rightarrow Y$ such that 
    \begin{equation}
        d_Y(f(x),f(x^{'}))=d_X(x,x^{'})
    \end{equation}
    for all $x,x^{'}\in X$ is called isometry between $X$ and $Y$. 
\end{definition}

\begin{definition}
    \emph{Isometry} is a distance-preserving transformation between metric spaces, usually assumed to be bijective.
\end{definition}

\begin{example}
    Conisder the sets $\mathbb{R}^2$ and $\mathbb{C}$, both with Euclidean metric. Then, the map $f: \mathbb{C}\rightarrow \mathbb{R}^2$ defined by $f(a+bi)=(a,b)$, which is isometry. 
\end{example}

\begin{example}
    Is $f(x)=x^2$ an isometry on Euclidean distance?
    $d(3,4)=\sqrt{9+16}=5$ but $d(f(3),f(4))=d(9,16)=\sqrt{81+256}=\sqrt{337}$
\end{example}

\begin{theorem}
    Let $(X,d)$ and $(Y,\rho)$ be metric spaces. A function $f:X\rightarrow Y$ is continuous if and only if $f^{-1}(U)$ is open in $X$ whenever $U$ is open in $Y$.
\end{theorem}
\begin{proof}
    Suppose first that $f$ is continuous
and that $U$ is open in $Y$. If $x\in f^{-1}(U)$, then
we can find a $y\in U$ with $f(x)=y$. Since $U$ is
open in $Y$, we can find an $\epsilon>0$ such that
\[z\in U\ \text{whenever $\rho(y,z)<\epsilon$}.\]
Since $f$ is continuous, we can find a $\delta>0$
such that
\[\rho(y,f(w))=\rho(f(x),f(w))<\epsilon
\ \text{whenever $d(x,w)<\delta$}.\]
Thus 
\[f(w)\in U\ \text{whenever $d(x,w)<\delta$}.\]
In other words,
\[w\in f^{-1}(U)\ \text{whenever $d(x,w)<\delta$}.\]
We have shown that $f^{-1}(U)$ is open.

We now seek the converse result. Suppose that
$f^{-1}(U)$ is open in $X$ whenever $U$ is open in $Y$.
Suppose $x\in X$ and $\epsilon>0$. We know that the
open ball
\[B(f(x),\epsilon)=\{y\in Y\,:\, \rho(f(x),y)<\epsilon\}\]
is open. Thus
$x\in f^{-1}\big(B(f(x),\epsilon)\big)$ and
$f^{-1}\big(B(f(x),\epsilon)\big)$ is open. It follows that
there is a $\delta>0$ such that
\[w\in f^{-1}\big(B(f(x),\epsilon)\big)\ \text{whenever $d(x,w)<\delta$},\]
so, in other words,
\[\rho(f(x),f(w))<\epsilon\ \text{whenever $d(x,w)<\delta$}.\]
Thus $f$ is continuous.
\end{proof}

\begin{theorem}
    If $(X,d)$, $(Y,\rho)$, $(Z,\sigma)$ are metric spaces and $g:X\rightarrow Y$, $f:Y\rightarrow Z$ are continuous, then so is the composition $f*g$
\end{theorem}
\begin{proof}
    If $U$ is open in $Z$, then, by continuity, $f^{-1}(U)$ is open in $Y$ and so, by continuity, $(fg)^{-1}(U)=g^{-1}(f^{-1}(U))$ is open in $X$. Thus $f*g$ is continuous. 
\end{proof}
    
\end{document}