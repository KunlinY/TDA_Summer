\documentclass[12pt,a4paper]{article}
\usepackage{amsmath,amsxtra,amsthm,amssymb,makeidx,graphics,graphicx}
\theoremstyle{plain}
\newtheorem{theorem}{Theorem}[section]
\newtheorem{lemma}[theorem]{Lemma}
\newtheorem{question}[theorem]{Question}
\newtheorem{exercise}[theorem]{Exercise}
\newtheorem{example}[theorem]{Example}
\newtheorem{definition}[theorem]{Definition}
\newtheoremstyle{citing}{3pt}{3pt}{\itshape}{0pt}{\bfseries}%
{.}{ }{\thmnote{#3}}\theoremstyle{citing}\newtheorem*{varthm}{}

\begin{document}

\title{Tutorial on Topological Data Analysis}
\author{Kunlin}
\maketitle

\tableofcontents

\begin{abstract}
    This is for the first tutorial of topological data analysis. 
\end{abstract}

\section{Tutorial on \LaTeX}

This is a \textbf{bold text}. \\
This is a \underline{underlined text}. \\
This is a \textit{italic text}. 

This could  also be \underline{\textbf{\textit{combined}}}.

If you don't want to choose manually, you could use \emph{emphsize}. 

\textit{\emph{emphsize} under italic text}. 
\textbf{\emph{emphasize} under bold text}. 

\begin{itemize}
    \item List one
    \item List two
\end{itemize}

\begin{enumerate}
    \item Number one
    \item Number two
\end{enumerate}

Metric: (i) $d(x,y) \geq 0$ for all $x,\,y \in X$.

\begin{equation}
    d(x,y) = 
    \begin{cases}
        0000000&\text{if $x=y$},\\
        1&\text{if $x\neq y$},
    \end{cases}
\end{equation}

\section{Metric Space}
Definition: Let $X$ be a set and $d:X^{2}\rightarrow{\mathbb R}$ a function with the follow properties:

(i) $d(x,y) \geq 0$ for all $x,\,y\in X$.

(ii) $d(x,y)=0$ if only if $x=y$. 

(iii) $d(x,y)=d(y,x)$ for all $x, \,y \in X$. 

(iv) $d(x,y)+d(y,z) \geq d(x,z)$ for all $x,\,y,\,z\in X$. (This is called the \emph{triangle inequality}).

Then we say that $d$ is a \emph{metric} on $X$ and that $(X,d)$ is a \emph{metric space}. 

Take away:

(i) Distances are always positive. 

(ii) Two points are zero distance part if and only if they are the same point.

(iii) The distance form $A$ to $B$ is the same as the distance from $B$ to $A$. 

(iv) The distance form $A$ to $B$ via $C$ is at least as great as the distance from $A$ to $B$ directly. 

\begin{exercise}
    If $d: X^2 \rightarrow \mathbb{R}$ is a function with the following properties: \\
    (ii) $d(x,y)=0$ if and only if $x=y$\\
    (iii) $d(x,y) = d(y,x)$ for all $x,y \in X$\\
    (iv)$d(x,y) + d(y,z) \geq d(x,z)$ for all $x,y,z\in X$\\
prove that $d$ is a metric on $X$ \\
$[$ Thus condition (i) of the definition is redundant. $]$
\end{exercise}

\noindent\emph{Solution.}  Setting $z=x$ in condition (iv):
\begin{equation}
    \begin{split}
        d(x,y) + d(y,x) &\geq d(x,x) \\
        2d(x,y)=d(x,y) + d(y,x) &\geq d(x, x) = 0 \\ 
        2d(x,y) &\geq 0 \\
        d(x,y) &\geq 0
    \end{split}
\end{equation}

\begin{definition}
    A \emph{metric} on the set $X$ is a function $d$: $X\times X\rightarrow [0,\inf)$ such that the following conditions are satisified for all $x,y,z\in X$:\\
(M1) Positive property: $d(x,y)=0$ if and only if $x=y$\\
(M2) Symmetry property: $d(x,y)=d(y,x)$\\
(M3) Triangle inequality: $d(x,z) \leq d(x,y) + d(y,z)$
\end{definition}

\begin{example}
    The set $X=\mathbb{R}$ with $d(x,y)=|x-y|$, the absolute value of the difference of $x-y$, prove that $d(x,y)$ is a metric on $X$. 
\end{example}
\noindent\emph{Solution.} M1 and M2 are obvisous. \\
For M3: 
\begin{equation}
    d(x,y) = |x-y| = |x-z+z-y| \leq |x-z| + |z-y| = d(x,z) + d(z,y)
\end{equation}

\begin{example}
    \label{manhattan}
    Let $X=\mathbb{R}^n$ and let $d:\mathbb{R}^n\times \mathbb{R}^n\rightarrow [0,\inf)$ be defined by:
    \begin{equation}
        d((x_1,x_2\cdots x_n),(y_1,y_2\cdots y_n)) = \sum^n_{i=1}|x_i-y_i|
    \end{equation}
    prove that $d(x,y)$ is a metric on $X$. 
\end{example}
\noindent\emph{Solution.}
\begin{equation}
    \begin{split}
        &d((x_1,x_2\cdots x_n),(y_1,y_2\cdots y_n))\\
        &= \sum^n_{i=1}|x_i-y_i| \\
        &= \sum^n_{i=1}|x_i-z_i + z_i - y_i|\\
        &\leq \sum^n_{i=1}|x_i-z_i| +| z_i - y_i| \\
        &=d((x_1,x_2\cdots x_n),(z_1,z_2\cdots z_n))+d((z_1,z_2\cdots z_n),(y_1,y_2\cdots y_n))
    \end{split}
\end{equation}

\begin{example}
    The \emph{Euclidean metric} on $\mathbb{R}^n$ is defined by formula:
    \begin{equation}
        d((x_1,x_2\cdots x_n),(y_1,y_2\cdots y_n))=\sqrt{\sum^n_{i=1}(x_i-y_i)^2}
    \end{equation}
    for each $(x_1,x_2\cdots x_n), (y_1,y_2\cdots y_n) \in \mathbb{R}^n$, prove that $d(x,y)$ is a metric on $X$. 
\end{example}
\noindent\emph{Solution.}
Let $x=(x_1,x_2\cdots x_n)$, $y=(y_1,y_2\cdots y_n)$, $z=(z_1,z_2\cdots z_n)$. \\
Let $r_i=x_i-z_i$ and $s_i=z_i-y_i$, we need to prove that 
\begin{equation}
    \begin{split}
        d(x,y)&=\sqrt{\sum^n_{i=1}(r_i+s_i)^2} \\
        &\leq\sqrt{\sum^n_{i=1}(r_i)^2}+\sqrt{\sum^n_{i=1}(s_i)^2} \\
        &=d(x,z) + d(z,y)
    \end{split}
\end{equation}

Since both sides of the inequality are positive. By squaring the above, it is equivalent to prove that:
\begin{equation}
    \begin{split}
    \sum^n_{i=1}(r_i+s_i)^2 &\leq \sum^n_{i=1}r_i^2 + \sum^n_{i=1}s_i^2 + 2\sqrt{\sum^n_{i=1}(r_i)^2}\sqrt{\sum^n_{i=1}(s_i)^2} \\
    (\sum^n_{i=1}r_is_i)^2 &\leq (\sum^n_{i=1}(s_i)^2)(\sum^n_{i=1}(r_i)^2)
    \end{split}
\end{equation}
The above could be derived from Cauchy-Schwartz inequality. 

\begin{example}
    There is also a Manhattan distance on $\mathbb{R}^n$:
    \begin{equation}
        d(x,y)=|y_1-x_1|+\cdots|y_n-x_n|
    \end{equation}
    Prove that it is also a metric. 
\end{example}
\noindent\emph{Solution.}
Same as example \ref{manhattan}

\begin{example}
    Consider box metric on $\mathbb{R}^n$:
    \begin{equation}
        d(x,y)=max\{|x_i-y_i|\}
    \end{equation}
    Prove that it is also a metric. 
\end{example}
\noindent\emph{Solution.}
Let $x=(x_1,\cdots,x_n)$, $y=(y_1,\cdots,y_n)$ and $z=(z_1,\cdots,z_n) \in \mathbb{R}^n$. Then, for each $i=1,\cdots,n$.
\begin{equation}
    |x_i-y_i| = |(x_i-z_i)+(z_i-y_i)| \leq |x_i-z_i| + |z_i=y_i| \leq d(x,z) + d(z,y)
\end{equation} 

\begin{example}
    Let $X$ be any set. The discrete metric on $X$ is defined by 
    \begin{equation}
        d(x,y)=
    \begin{cases}
    0&\text{if $x=y$},\\
    1&\text{if $x\neq y$},
    \end{cases}
    \end{equation}
    Prove that it is also a metric. 
\end{example}
\noindent\emph{Solution.}
Let $x,y,z\in X$. If $x=y$, then $d(x,y)=0$, and there is nothing to check. \\
Suppose then taht $x\neq y$. Then, either $z=x$ or $z=y$ or $z\neq x,y$. Regardless the situation, we have then
\begin{equation}
    1=d(x,y) \; \text{and} \; 1 \leq d(x,z) + d(z,y) \leq 2
\end{equation}
and the triangle inequality hodls. 

\begin{example}
    Let $d_1$, $d_2$ and $d_{\infty}$ be the following metrics $\mathbb{R}^2$:
    \begin{itemize}
        \item $d_1((x_1,y_1),(x_2,y_2))=|x_1-x_2|+|y_1-y_2|$
        \item $d_2((x_1,y_1),(x_2,y_2))=\sqrt{(x_1-x_2)^2+(y_1-y_2)^2}$
        \item $d_{\infty}((x_1,y_1),(x_2,y_2))=max\{|x_1-x_2|,|y_1-y_2|\}$
    \end{itemize}
    Then, for each $x_1,y_1),(x_2,y_2)\in \mathbb{R}^2$, prove\\
    $\frac12d_1((x_1,y_1),(x_2,y_2))\leq \frac{1}{\sqrt{2}}d_2((x_1,y_1),(x_2,y_2)) \leq d_{\infty}((x_1,y_1),(x_2,y_2))$
\end{example}
\noindent\emph{Solution.}
By definition of the metric $d_2$,
\begin{equation}
    \begin{split}
    d_2((x_1,y_1),(x_2,y_2)) &= \sqrt{(x_1-x_2)^2+(y_1-y_2)^2} \\
    &\leq \sqrt{2max\{(x_1-x_2)^2, (y_1-y_2)^2\}} \\
    &=d_{\infty}((x_1,y_1),(x_2,y_2))
    \end{split}
\end{equation}
For $d_1$ and $d_2$, let $a=|x_1-x_2|$ and $b=|y_1-_2|$. Then
\begin{equation}
    (a+b)^2 \leq 2(a^2+b^2)
\end{equation}
which is obviously true. 
\end{document}